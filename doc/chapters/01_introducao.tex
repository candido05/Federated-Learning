\section{INTRODUÇÃO}

\subsection{Motivação}

O Aprendizado Federado tem se consolidado como uma abordagem promissora para treinamento distribuído de modelos de aprendizado de máquina, permitindo que dispositivos colaborem no desenvolvimento de modelos globais sem compartilhar dados sensíveis. Esta característica torna-se especialmente relevante no contexto atual, onde questões de privacidade e proteção de dados pessoais ganham cada vez mais importância (POR REFERENCIA), tanto do ponto de vista regulatório quanto ético. No entanto, a maioria das pesquisas em FL concentra-se em redes neurais profundas, que demandam recursos computacionais significativos, incluindo grande capacidade de processamento, memória e, frequentemente, unidades de processamento gráfico especializadas (POR REFERENCIA).

Modelos baseados em árvore, como XGBoost, LightGBM e CatBoost, apresentam-se como alternativas mais leves e eficientes para diversos problemas de aprendizado de máquina. Estes algoritmos são capazes de serem treinados em hardwares mais modestos, sem a necessidade de GPUs dedicadas ou grandes volumes de memória RAM. Esta característica torna-os particularmente adequados para ambientes distribuídos heterogêneos, onde dispositivos podem possuir capacidades computacionais limitadas, como dispositivos móveis, sistemas embarcados ou computadores pessoais convencionais (POR REFERENCIA). A leveza computacional destes modelos não apenas reduz as barreiras de entrada para participação em sistemas federados, mas também amplia significativamente o espectro de dispositivos que podem contribuir para o treinamento colaborativo.

Além disso, a eficiência energética dos modelos baseados em árvore contribui para um treinamento mais sustentável. Em um cenário onde a pegada de carbono dos sistemas de aprendizado de máquina é crescentemente questionada, a redução do consumo energético durante o treinamento representa um avanço importante. Esta sustentabilidade é particularmente relevante em ambientes federados, onde múltiplos dispositivos treinam simultaneamente, potencialmente multiplicando o impacto ambiental. A adoção de modelos mais leves pode, portanto, contribuir para a viabilidade de sistemas FL em cenários reais, especialmente em regiões com infraestrutura energética limitada ou em aplicações que priorizam eficiência operacional.

Apesar destas vantagens, a comunicação entre clientes e servidor em sistemas FL representa um gargalo significativo. O overhead de rede, a latência na transmissão de parâmetros e a necessidade de sincronização frequente podem impactar negativamente o desempenho global do sistema (POR REFERENCIA). Neste contexto, a integração com Redes Definidas por Software emerge como uma solução potencial para otimizar a infraestrutura de comunicação, possibilitando gerenciamento dinâmico de tráfego e priorização inteligente de fluxos de dados (POR REFERENCIA).

\subsection{Objetivo}

Este trabalho tem como objetivo principal investigar a eficiência de modelos baseados em árvore, especificamente, XGBoost, LightGBM e CatBoost, em ambientes de Aprendizado Federado quando otimizados através do uso de Redes Definidas por Software (SDN). A escolha destes 3 algoritmos justifica-se por suas diferentes abordagens ao problema de gradient boosting, cada um com características particulares que podem apresentar comportamentos distintos em ambientes distribuídos.

||||| Especificamente, este estudo busca avaliar se a integração de SDN pode melhorar o desempenho destes modelos através de múltiplas dimensões. Primeiro, investiga-se a possibilidade de redução de overhead de comunicação mediante o gerenciamento inteligente de tráfego de rede, evitando congestionamentos e otimizando o uso de banda disponível. Segundo, analisa-se o potencial de otimização de latência através da priorização de fluxos críticos e seleção dinâmica de rotas de menor tempo de resposta. Terceiro, examina-se a viabilidade de implementação de políticas de Quality of Service (QoS) que priorizem inteligentemente o tráfego de rede entre diferentes clientes e o servidor central. |||||

A hipótese central que orienta esta pesquisa é que a combinação da leveza computacional dos modelos baseados em árvore com a capacidade de gerenciamento dinâmico de rede proporcionada por SDN pode resultar em sistemas FL significativamente mais eficientes, sustentáveis e escaláveis para aplicações práticas em diversos domínios. Esta hipótese fundamenta-se na premissa teórica e empiricamente fundamentada de que, ao reduzir simultaneamente os custos computacionais locais nos dispositivos clientes e os custos de comunicação na infraestrutura de rede distribuída, pode-se alcançar um sistema federado com desempenho superior ao obtido com abordagens convencionais que otimizam apenas uma destas dimensões. Mais especificamente, argumenta-se que a natureza menos intensiva em recursos dos modelos baseados em árvore, quando combinada com políticas inteligentes de roteamento e priorização de tráfego habilitadas por SDN, pode reduzir significativamente o tempo total de convergência do modelo global, minimizar o consumo de energia agregado do sistema, e melhorar a robustez do treinamento federado frente a condições adversas de rede, tais como alta latência, perda de pacotes, ou largura de banda limitada. Esta abordagem sinérgica representa uma contribuição metodológica importante para o campo do Aprendizado Federado, ao considerar de forma holística tanto os aspectos computacionais quanto os aspectos de infraestrutura de rede que influenciam o desempenho global do sistema.

Adicionalmente, este trabalho visa contribuir para o corpo de conhecimento em Aprendizado Federado ao explorar uma combinação ainda pouco investigada na literatura científica: modelos baseados em árvore otimizados com Redes Definidas por Software. Enquanto a maioria dos estudos existentes concentra-se primariamente em redes neurais profundas ou explora otimizações isoladas focando em apenas uma dimensão do problema (POR REFERENCIA), esta pesquisa propõe uma abordagem holística e integrada que considera simultaneamente tanto a eficiência computacional dos algoritmos de aprendizado quanto a eficiência de comunicação da infraestrutura de rede subjacente. Os resultados obtidos através desta investigação poderão orientar futuras implementações práticas de sistemas FL em ambientes com múltiplas restrições de recursos, sejam elas computacionais (processamento e memória), energéticas (consumo elétrico e autonomia de bateria), ou de infraestrutura de rede (largura de banda, latência e confiabilidade). Além disso, espera-se que este trabalho contribua para a identificação de melhores práticas e diretrizes de design para sistemas de Aprendizado Federado em ambientes reais, particularmente em contextos onde a heterogeneidade dos dispositivos participantes e a variabilidade das condições de rede representam desafios significativos para a convergência e qualidade do modelo global resultante. A abordagem metodológica adotada, combinando experimentação empírica com análise comparativa de diferentes algoritmos e estratégias de agregação, busca fornecer insights práticos que possam ser aplicados em cenários reais de produção, contribuindo assim para a transição do Aprendizado Federado de um paradigma predominantemente acadêmico para uma tecnologia madura e amplamente adotada na indústria.

\subsection{Estrutura da monografia}
Esta monografia está dividida da seguinte maneira: na Seção 2 é feita a revisão da literatura sobre aprendizado federado e redes SDN. Na Seção 3 são apresentados os fundamentos teóricos necessários para compreensão do trabalho. Na Seção 4 é apresentada a metodologia utilizada no desenvolvimento do trabalho. Na Seção 5 o dataset utilizado é descrito em detalhes. Na Seção 6 a configuração experimental é apresentada. Na Seção 7 os resultados obtidos são apresentados e analisados. Na Seção 8 é feita a discussão dos resultados. Na Seção 9 é feita a conclusão deste trabalho e trabalhos futuros são propostos.

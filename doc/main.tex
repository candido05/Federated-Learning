%%%%%%%%%%%%%%%%%%%%%%%%%%%%% TCC %%%%%%%%%%%%%%%%%%%%%%%%%%%%%%%%
%
% Template para TCC da Universidade Federal da Paraíba
%
% Adaptado para o projeto de Federated Learning com SDN
% Autor: Cândido Leandro de Queiroga Bisneto
% Orientador: Fernando Menezes Matos
%
%%%%%%%%%%%%%%%%%%%%%%%%%%%%%%%%%%%%%%%%%%%%%%%%%%%%%%%%%%%%%%%%%%%

\documentclass{tcc}

\begin{document}
\pagestyle{empty} %retira numeração da página
%Dados do TCC%
\author{Cândido Leandro de Queiroga Bisneto}
\title{Otimização de Modelos de Aprendizado Federado com Redes Definidas por Software}
\newcommand{\subtitulo}{}
\newcommand{\nomedocurso}{Ciência da Computação}
\newcommand{\titulobar}{Ciência da Computação}
\newcommand{\orientador}{Fernando Menezes Matos}
\newcommand{\profa}{Fernando Menezes Matos}
\newcommand{\profb}{[Nome do Professor B]}
\newcommand{\profc}{[Nome do Professor C]}
\newcommand{\insta}{Universidade Federal da Paraíba}
\newcommand{\instb}{Universidade Federal da Paraíba}
\newcommand{\instc}{Universidade Federal da Paraíba}
\newcommand{\coordenador}{[Nome do Coordenador]}
\newcommand{\departamento}{Ciência da Computação}


% CAPA
\begin{center}
\LARGE{\bf \thetitle}\\
\end{center}

\vspace{1em}

\begin{center}
\theauthor
\end{center}

\vfill

\begin{figure}[H]
\centering
\includegraphics[height=3cm]{figures/ci_logo.png}
\hspace{1cm}
\includegraphics[height=3cm]{figures/ufpb_logo.png}
\end{figure}

\begin{center}
CENTRO DE INFORMÁTICA \\
UNIVERSIDADE FEDERAL DA PARAÍBA
\end{center}

\vspace{2in}

\begin{center}
João Pessoa, 2025
\end{center}
\afterpage{\blankpage \addtocounter{page}{1}}

% FOLHA DE ROSTO
\newpage
\begin{center}
\theauthor
\end{center}
\vspace{3in}
\begin{center}
\LARGE{\thetitle}\\
\end{center}

\vspace{2in}

\begin{flushright}
Monografia apresentada ao curso \nomedocurso \\ do Centro de Informática, da Universidade Federal da Paraíba, \\ como requisito para a obtenção do grau de Bacharel em \titulobar
\\
\vspace{0.2in}

Orientador: \orientador

\end{flushright}

\vfill
\begin{center}
Dezembro de 2025
\end{center}

\newpage

% PÁGINA EM BRANCO (para ficha catalográfica no verso)
$ $
\vfill

\newpage

% PÁGINA DE APROVAÇÃO
\begin{figure}[H]
\centering
\includegraphics[height=3cm]{figures/ci_logo.png}
\hspace{1cm}
\includegraphics[height=3cm]{figures/ufpb_logo.png}
\end{figure}

\begin{center}
CENTRO DE INFORMÁTICA \\
UNIVERSIDADE FEDERAL DA PARAÍBA
\end{center}

\vspace{0.05in}

Trabalho de Conclusão de Curso de \nomedocurso \ intitulado \textit{\bf \em \thetitle} de autoria de \theauthor, aprovada pela banca examinadora constituída pelos seguintes professores: \\

\vspace{0.7in}

\hrule
\noindent Prof. Dr. \profa\\
\insta\\

\vspace{0.25in}

\hrule
\noindent Prof. \profb\\
\instb\\

\vspace{0.25in}

\hrule
\noindent Prof. \profc\\
\instc\\

\vspace{0.8in}

\hrule
\noindent Coordenador(a) do Curso de \departamento\\
\coordenador\\
CI/UFPB\\

\vfill

\begin{center}
João Pessoa, Dezembro de 2025
\end{center}

\afterpage{\blankpage \addtocounter{page}{1}}

% DEDICATÓRIA
\newpage
\section*{\centering{DEDICATÓRIA}}
[Texto da dedicatória - opcional]


% AGRADECIMENTOS
\newpage
\section*{\centering{AGRADECIMENTOS}}

[Texto dos agradecimentos]


% RESUMO
\newpage
\section*{\centering{RESUMO}}

[Inserir resumo em português - um único parágrafo informativo contendo: objetivos do trabalho, justificativa e resultados alcançados. Máximo 200 palavras.]

{\bf Palavras-chave:} Aprendizado Federado, Redes Definidas por Software, XGBoost, LightGBM, CatBoost, Flower Framework, Privacidade de Dados.


% ABSTRACT
\newpage
\section*{\centering{ABSTRACT}}

[Insert abstract in English - one informative paragraph containing: work objectives, justification, and results achieved. Maximum 200 words.]

{\bf Keywords:} Federated Learning, Software-Defined Networking, XGBoost, LightGBM, CatBoost, Flower Framework, Data Privacy.


% LISTA DE FIGURAS
\newpage
\renewcommand{\listfigurename}{\centering LISTA DE FIGURAS}
\listoffigures

% LISTA DE TABELAS
\newpage
\renewcommand{\listtablename}{\centering LISTA DE TABELAS}
\listoftables

% LISTA DE ABREVIATURAS
\newpage
\section*{\centering{LISTA DE ABREVIATURAS}}

API - Application Programming Interface

AUC - Area Under the Curve

CART - Classification and Regression Trees

CPU - Central Processing Unit

DP - Differential Privacy

FL - Federated Learning

GBDT - Gradient Boosting Decision Trees

GPU - Graphics Processing Unit

gRPC - Google Remote Procedure Call

HTTP - Hypertext Transfer Protocol

IID - Independent and Identically Distributed

LGPD - Lei Geral de Proteção de Dados

ML - Machine Learning

non-IID - Non-Independent and Identically Distributed

QoS - Quality of Service

ROC - Receiver Operating Characteristic

SDN - Software-Defined Networking

TCC - Trabalho de Conclusão de Curso

TCP - Transmission Control Protocol

UFPB - Universidade Federal da Paraíba

UDP - User Datagram Protocol


% SUMÁRIO
\newpage
\pagestyle{plain} %mostra numeração da página%
\tableofcontents

% CAPÍTULOS
\newpage
\section{INTRODUÇÃO}

\subsection{Motivação}

\subsection{Objetivo}

\subsection{Estrutura da monografia}
Esta monografia está dividida da seguinte maneira: na segunda seção é feita a revisão da literatura sobre aprendizado federado e redes SDN. Na terceira seção são apresentados os fundamentos teóricos necessários para compreensão do trabalho. Na quarta seção é apresentada a metodologia utilizada no desenvolvimento do trabalho. Na quinta seção o dataset utilizado é descrito em detalhes. Na sexta seção a configuração experimental é apresentada. Na sétima seção os resultados obtidos são apresentados e analisados. Na oitava seção é feita a discussão dos resultados. Na nona seção é feita a conclusão deste trabalho e trabalhos futuros são propostos.

\section{REVISÃO DE LITERATURA}

\subsection{Trabalhos Relacionados sobre Aprendizado Federado}

\subsection{Trabalhos sobre Modelos Baseados em Árvores}

\subsection{Trabalhos sobre SDN e Otimização de Rede}

\subsection{Análise Comparativa dos Trabalhos}

\section{FUNDAMENTOS TEÓRICOS}

\subsection{Aprendizado Federado}

\subsubsection{Definição e Características}

\subsubsection{Estratégias de Agregação}

\subsection{Modelos Baseados em Árvores de Decisão}

\subsubsection{XGBoost}

\subsubsection{LightGBM}

\subsubsection{CatBoost}

\subsection{Redes Definidas por Software (SDN)}

\subsubsection{Arquitetura SDN}

\subsubsection{OpenFlow}

\subsubsection{Aplicações em Aprendizado de Máquina}

\subsection{Framework Flower}

\subsubsection{Arquitetura do Flower}

\subsubsection{Implementação de Clientes e Servidores}

\section{METODOLOGIA}

\subsection{Visão Geral da Solução Proposta}

\subsection{Arquitetura do Sistema}

\subsubsection{Componentes do Sistema}

\subsubsection{Fluxo de Comunicação}

\subsection{Implementação dos Modelos Federados}

\subsubsection{Adaptação de XGBoost para FL}

\subsubsection{Adaptação de LightGBM para FL}

\subsubsection{Adaptação de CatBoost para FL}

\subsection{Integração com SDN}

\subsubsection{Configuração da Rede SDN}

\subsubsection{Políticas de QoS}

\subsubsection{Monitoramento de Tráfego}

\subsection{Tecnologias Utilizadas}

\begin{table}[H]
\centering
\begin{tabularx}{\textwidth}{|l|X|}
\hline
\bf Componente & \bf Tecnologia \\ \hline
Framework FL & Flower 1.6+ \\ \hline
Modelos & XGBoost, LightGBM, CatBoost \\ \hline
Linguagem & Python 3.9+ \\ \hline
SDN Controller & [A definir] \\ \hline
Comunicação & gRPC \\ \hline
\end{tabularx}
\caption{Tecnologias utilizadas no projeto}
\label{tab:tecnologias}
\end{table}

\section{DATASET E PREPARAÇÃO DOS DADOS}

\subsection{Origem e Geração do Dataset}

Para avaliar o desempenho dos algoritmos de Aprendizado Federado neste trabalho, foi utilizado um dataset de comportamento de veículos gerado através de simulação de tráfego veicular realista. O dataset original foi criado utilizando um simulador de tráfego especializado que implementa o modelo de mobilidade IDM (Intelligent Driver Model), um modelo amplamente reconhecido na literatura de simulação de tráfego que captura comportamentos realistas de condução, incluindo aceleração, desaceleração, mudanças de faixa e interações entre veículos. A escolha de dados sintéticos gerados por simulação, ao invés de dados reais de tráfego, permite controle preciso sobre parâmetros experimentais, reprodutibilidade dos experimentos através de seeds aleatórias fixas, e evita questões de privacidade e regulamentações associadas a dados reais de veículos. O dataset completo está armazenado em:

\begin{verbatim}
dataset_fl/dataset/dataset_K400_seed42/
\end{verbatim}

\subsubsection{Configuração da Simulação}

A simulação que gerou o dataset foi configurada com um conjunto detalhado de parâmetros que definem o ambiente de tráfego, características dos veículos e duração da observação. Estes parâmetros foram armazenados em um arquivo de configuração \texttt{simulation\_config.json} para garantir reprodutibilidade e documentação completa dos experimentos. A configuração escolhida visa simular um cenário de rodovia realista com densidade moderada de tráfego e condições operacionais típicas:

\begin{itemize}
    \item \textbf{Número de veículos (K)}: 400 veículos simulados
    \item \textbf{Rodovia}: 88 km de extensão com 3 faixas
    \item \textbf{Tempo de simulação}: 3.600 segundos (1 hora)
    \item \textbf{Intervalo de amostragem}: 1 segundo
    \item \textbf{Seed aleatória}: 42 (para reprodutibilidade)
    \item \textbf{Modelo de mobilidade}: IDM com parâmetros:
    \begin{itemize}
        \item Velocidade desejada: 25 m/s (90 km/h)
        \item Distância de segurança: 2 m
        \item Tempo de reação: 1,5 s
        \item Aceleração máxima: 1,0 m/s²
        \item Desaceleração confortável: 1,5 m/s²
    \end{itemize}
\end{itemize}

\subsubsection{Estrutura Original dos Dados (JSON)}

Os dados brutos da simulação foram salvos em formato JSON estruturado, com 1 arquivo independente por veículo. Esta organização facilita o particionamento natural dos dados por veículo no contexto do Aprendizado Federado, onde cada arquivo representa os dados coletados por um cliente individual (veículo). O formato JSON foi escolhido por sua legibilidade, facilidade de parsing e flexibilidade para armazenar estruturas de dados heterogêneas. A estrutura de diretórios é a seguinte:

\begin{verbatim}
private_datasets/
├── veh_0.json
├── veh_1.json
├── veh_2.json
...
└── veh_399.json
\end{verbatim}

Cada arquivo JSON contém uma lista de observações temporais do veículo, onde cada observação possui:
\begin{itemize}
    \item \textbf{features}: Array com 22 valores numéricos (estado do veículo e vizinhança)
    \item \textbf{label}: Classe de comportamento (0, 1 ou 2)
\end{itemize}

Exemplo de estrutura JSON:
\begin{verbatim}
[
  {
    "features": [21.18, 0, 21.23, 1, 385.34, 20.86, ...],
    "label": 1
  },
  {
    "features": [23.25, 0, 23.35, 1, 262.89, 25.26, ...],
    "label": 2
  },
  ...
]
\end{verbatim}

\subsection{Transformação para Formato Tabular}

Os dados JSON foram transformados em formato CSV tabular para facilitar o treinamento dos modelos baseados em árvore de decisão (XGBoost, LightGBM e CatBoost), que operam de forma mais eficiente com estruturas tabulares do que com formatos hierárquicos como JSON. Este processo de transformação consolidou todos os 400 arquivos JSON individuais em 2 arquivos CSV estruturados (treino e validação), criando uma representação matricial dos dados onde cada linha corresponde a uma observação temporal de um veículo e cada coluna representa uma feature numérica. A transformação preserva todas as informações relevantes para o treinamento, incluindo metadados como identificadores de veículos necessários para o particionamento federado:

\begin{table}[H]
\centering
\begin{tabularx}{\textwidth}{|l|X|}
\hline
\bf Característica & \bf Valor \\ \hline
Nome & Dataset de Comportamento de Veículos \\ \hline
Formato original & 400 arquivos JSON (1 por veículo) \\ \hline
Formato processado & 2 arquivos CSV tabulares \\ \hline
Amostras de treino & 115.511 observações \\ \hline
Amostras de validação & 28.878 observações \\ \hline
Veículos & 400 veículos únicos \\ \hline
Número de features & 22 features numéricas \\ \hline
Classes & 3 classes de comportamento (0, 1, 2) \\ \hline
Tipo de problema & Classificação multi-classe \\ \hline
Simulador & IDM (Intelligent Driver Model) \\ \hline
Seed & 42 (para reprodutibilidade) \\ \hline
\end{tabularx}
\caption{Metadados do dataset de veículos utilizado nos experimentos}
\label{tab:dataset_metadata}
\end{table}

O processo de transformação JSON → CSV consolidou os dados de múltiplos arquivos JSON em 2 arquivos CSV estruturados:

\begin{itemize}
    \item \textbf{Treino}: \texttt{dataset\_all\_vehicles.csv} (115.511 amostras de 400 veículos)
    \item \textbf{Validação}: \texttt{dataset\_validation\_all\_vehicles.csv} (28.878 amostras)
\end{itemize}

\textbf{Processo de conversão}:
\begin{enumerate}
    \item Leitura de cada arquivo JSON (\texttt{veh\_*.json}) contendo observações temporais
    \item Extração dos arrays \texttt{features} e valores \texttt{label} de cada observação
    \item Adição de coluna \texttt{vehicle\_id} para identificar a origem dos dados
    \item Concatenação de todas as observações em uma única estrutura tabular
    \item Adição de cabeçalhos descritivos para cada feature
    \item Divisão em conjuntos de treino e validação
    \item Exportação para formato CSV com codificação UTF-8
\end{enumerate}

\subsection{Descrição das Features}

O dataset contém 22 features numéricas cuidadosamente selecionadas que descrevem de forma abrangente o estado de um veículo (ego-vehicle) e sua vizinhança imediata no ambiente de tráfego simulado. Estas features capturam informações espaciais (posições e distâncias relativas), cinemáticas (velocidades absolutas e relativas), e contextuais (presença de veículos vizinhos, faixa de rodagem) que são essenciais para caracterizar o comportamento do veículo e permitir classificação precisa. As features foram organizadas em categorias lógicas baseadas em sua natureza e na região espacial que descrevem em relação ao ego-vehicle, facilitando a interpretação e análise posterior:

\begin{table}[H]
\centering
\begin{tabularx}{\textwidth}{|l|X|l|}
\hline
\bf Feature & \bf Descrição & \bf Tipo \\ \hline
\multicolumn{3}{|c|}{\textbf{Estado do Ego-Vehicle}} \\ \hline
ego\_speed & Velocidade do veículo principal & float32 \\ \hline
ego\_lane & Faixa atual do veículo (0, 1, 2) & float32 \\ \hline
ego\_desired\_speed & Velocidade desejada pelo veículo & float32 \\ \hline
\multicolumn{3}{|c|}{\textbf{Veículo à Frente}} \\ \hline
front\_exists & Indica se existe veículo à frente (0/1) & float32 \\ \hline
front\_distance & Distância para o veículo à frente & float32 \\ \hline
front\_speed & Velocidade do veículo à frente & float32 \\ \hline
\multicolumn{3}{|c|}{\textbf{Veículo à Esquerda (Frente)}} \\ \hline
left\_front\_exists & Indica se existe veículo na frente-esquerda & float32 \\ \hline
left\_front\_distance & Distância para o veículo frente-esquerda & float32 \\ \hline
left\_front\_speed & Velocidade do veículo frente-esquerda & float32 \\ \hline
\multicolumn{3}{|c|}{\textbf{Veículo à Esquerda (Trás)}} \\ \hline
left\_rear\_exists & Indica se existe veículo atrás-esquerda & float32 \\ \hline
left\_rear\_distance & Distância para o veículo atrás-esquerda & float32 \\ \hline
left\_rear\_speed & Velocidade do veículo atrás-esquerda & float32 \\ \hline
\multicolumn{3}{|c|}{\textbf{Veículo à Direita (Frente)}} \\ \hline
right\_front\_exists & Indica se existe veículo na frente-direita & float32 \\ \hline
right\_front\_distance & Distância para o veículo frente-direita & float32 \\ \hline
right\_front\_speed & Velocidade do veículo frente-direita & float32 \\ \hline
\multicolumn{3}{|c|}{\textbf{Veículo à Direita (Trás)}} \\ \hline
right\_rear\_exists & Indica se existe veículo atrás-direita & float32 \\ \hline
right\_rear\_distance & Distância para o veículo atrás-direita & float32 \\ \hline
right\_rear\_speed & Velocidade do veículo atrás-direita & float32 \\ \hline
\multicolumn{3}{|c|}{\textbf{Features Derivadas}} \\ \hline
speed\_diff\_front & Diferença de velocidade com veículo à frente & float32 \\ \hline
speed\_diff\_left\_front & Diferença de velocidade com veículo frente-esquerda & float32 \\ \hline
speed\_diff\_right\_front & Diferença de velocidade com veículo frente-direita & float32 \\ \hline
speed\_diff\_left\_rear & Diferença de velocidade com veículo atrás-esquerda & float32 \\ \hline
\multicolumn{3}{|c|}{\textbf{Metadados}} \\ \hline
vehicle\_id & Identificador único do veículo & int \\ \hline
label & Classe alvo (0, 1 ou 2) & int \\ \hline
\end{tabularx}
\caption{Features do dataset de comportamento de veículos}
\label{tab:dataset_features}
\end{table}

As features capturam informações espaciais (distâncias), cinemáticas (velocidades) e relacionais (diferenças de velocidade) que são essenciais para a classificação do comportamento veicular.

\subsection{Pré-processamento}

O pré-processamento dos dados foi implementado de forma modular e reutilizável na classe \texttt{DataProcessor} do módulo \texttt{common/data\_processing.py}, seguindo um pipeline estruturado de transformações que prepara os dados brutos para o treinamento federado. Este pipeline inclui etapas de carregamento, limpeza, normalização, detecção de metadados e particionamento, todas executadas de forma automática e configurável. A implementação segue boas práticas de machine learning, como separação clara entre conjuntos de treino e validação, normalização consistente aplicada antes do particionamento, e preservação de informações necessárias para o contexto federado (como identificadores de veículos).

\subsubsection{Carregamento e Leitura dos Dados}

Os arquivos CSV são carregados utilizando \texttt{pandas.read\_csv()}, que realiza:
\begin{itemize}
    \item Leitura do arquivo CSV com cabeçalho
    \item Parsing automático dos tipos de dados (float32 para features, int para labels)
    \item Carregamento em memória como DataFrame do pandas
\end{itemize}

\subsubsection{Tratamento de Metadados}

O dataset contém 2 colunas especiais que não são features preditivas, mas sim metadados essenciais para o Aprendizado Federado, e portanto recebem tratamento diferenciado durante o pré-processamento:
\begin{itemize}
    \item \textbf{vehicle\_id}: Extraída e armazenada separadamente para particionamento por veículo no Aprendizado Federado, removida da matriz de features antes do treinamento
    \item \textbf{label}: Coluna alvo com as classes (0, 1, 2), separada das features e convertida para tipo \texttt{int}
\end{itemize}

Implementação (\texttt{common/data\_processing.py}, método \texttt{\_load\_csv}):
\begin{verbatim}
if "vehicle_id" in df.columns:
    vehicle_ids = df["vehicle_id"].values
    cols_to_drop.append("vehicle_id")

if "label" in df.columns:
    y = df["label"].values.astype(int)
    cols_to_drop.append("label")

X = df.drop(columns=cols_to_drop).values.astype(np.float32)
\end{verbatim}

\subsubsection{Normalização e Padronização}

Todos os dados numéricos foram normalizados utilizando \texttt{StandardScaler} do scikit-learn, um transformador que aplica padronização Z-score (também conhecida como standardization), convertendo cada feature para ter média zero e variância unitária. Esta normalização é crucial para modelos baseados em árvore quando as features possuem escalas muito diferentes, e garante que nenhuma feature domine a métrica de divisão apenas por ter valores absolutos maiores. A transformação Z-score é definida matematicamente como:

\begin{equation}
z = \frac{x - \mu}{\sigma}
\end{equation}

onde $\mu$ é a média e $\sigma$ é o desvio padrão das features.

O procedimento de normalização seguiu rigorosamente 3 boas práticas fundamentais de machine learning para evitar data leakage e garantir avaliação justa do modelo:
\begin{enumerate}
    \item \textbf{Fit no treino}: O scaler foi ajustado (\texttt{fit}) apenas nos dados de treino
    \item \textbf{Transform em ambos}: A transformação foi aplicada tanto no treino quanto na validação
    \item \textbf{Antes do particionamento}: A normalização ocorreu antes da distribuição dos dados entre clientes
\end{enumerate}

Implementação no código (\texttt{common/data\_processing.py}, método \texttt{load\_and\_prepare\_data}):
\begin{verbatim}
# Inicializar scaler
self.scaler = StandardScaler()

# Fit apenas nos dados de treino
self.scaler.fit(X_train_all)

# Transform em treino e validação
X_train_all = self.scaler.transform(X_train_all)
X_test_all = self.scaler.transform(X_test_all)
\end{verbatim}

\subsubsection{Detecção de Número de Classes}

O número de classes no problema de classificação é detectado automaticamente através da análise dos valores únicos presentes na coluna \texttt{label} do conjunto de treinamento, utilizando a função \texttt{np.unique()} do NumPy que retorna um array com todos os valores distintos encontrados:
\begin{verbatim}
num_classes = len(np.unique(y_train_all))
\end{verbatim}

Esta detecção automática permite que os algoritmos (XGBoost, LightGBM, CatBoost) configurem corretamente:
\begin{itemize}
    \item Função objetivo (binária vs. multi-classe)
    \item Número de classes para softmax
    \item Métricas de avaliação apropriadas
\end{itemize}

Para o dataset de veículos, o número detectado é 3 classes, configurando automaticamente os modelos para classificação multi-classe com função objetivo \texttt{multi:softprob} (XGBoost) ou equivalentes.

\subsection{Particionamento dos Dados para Federated Learning}

\subsubsection{Particionamento por Veículo}

A estratégia principal de particionamento utilizada neste trabalho é o \textbf{particionamento por veículo} (vehicle-based partitioning), onde todos os dados coletados de um mesmo veículo durante a simulação são alocados exclusivamente a um único cliente, mantendo a integridade e localidade dos dados por dispositivo. Esta abordagem é significativamente mais realista e apropriada para cenários de Aprendizado Federado veicular do que particionamentos aleatórios (IID), pois melhor representa a natureza distribuída e descentralizada dos dados em aplicações reais, onde cada veículo (cliente) naturalmente coleta e mantém suas próprias observações localmente. As principais vantagens desta estratégia incluem:

\begin{itemize}
    \item Simula dados reais onde cada veículo (cliente) coleta suas próprias observações
    \item Cria naturalmente partições non-IID, já que diferentes veículos podem ter comportamentos distintos
    \item Preserva a correlação temporal entre observações do mesmo veículo
\end{itemize}

\textbf{Algoritmo de particionamento (5 etapas)}:
\begin{enumerate}
    \item Identificar todos os veículos únicos no dataset (coluna \texttt{vehicle\_id})
    \item Embaralhar aleatoriamente os IDs dos veículos (seed=42)
    \item Dividir os veículos em $N$ grupos de tamanho aproximadamente igual
    \item Atribuir cada grupo de veículos a um cliente
    \item Cada cliente recebe todas as amostras dos veículos do seu grupo
\end{enumerate}

Implementação (\texttt{common/data\_processing.py}):
\begin{verbatim}
unique_vehicles = np.unique(vehicle_ids_train)
shuffled_vehicles = np.random.permutation(unique_vehicles)
vehicles_per_client = num_vehicles // self.num_clients

for client_id in range(self.num_clients):
    client_vehicle_ids = shuffled_vehicles[start_idx:end_idx]
    mask = np.isin(vehicle_ids_train, client_vehicle_ids)
    client_X = X_train_all[mask]
    client_y = y_train_all[mask]
\end{verbatim}

\textbf{Características da distribuição}:
\begin{itemize}
    \item \textbf{Total de dados}: 115.511 amostras distribuídas entre clientes
    \item \textbf{Distribuição de amostras}: Variável por cliente (depende do número de observações por veículo)
    \item \textbf{Distribuição de classes}: Naturalmente non-IID, refletindo comportamentos específicos de cada veículo
    \item \textbf{Validação}: Conjunto centralizado de 28.878 amostras utilizado pelo servidor
\end{itemize}

\subsubsection{Particionamento IID (Fallback)}

Como estratégia alternativa de fallback, o sistema também implementa particionamento IID (Independent and Identically Distributed) tradicional em 3 etapas sequenciais, utilizado automaticamente quando a coluna \texttt{vehicle\_id} não está disponível no dataset ou quando deseja-se avaliar o desempenho em cenários ideais com distribuição uniforme de dados:

\begin{enumerate}
    \item Embaralhar todas as amostras aleatoriamente
    \item Dividir o dataset em $N$ partes iguais
    \item Atribuir cada parte a um cliente
\end{enumerate}

Esta estratégia garante distribuição uniforme de classes entre clientes, mas não é utilizada nos experimentos principais deste trabalho, pois o dataset possui informação de \texttt{vehicle\_id}.

\section{CONFIGURAÇÃO EXPERIMENTAL}

\subsection{Ambiente Computacional}

\begin{table}[H]
\centering
\begin{tabularx}{\textwidth}{|l|X|}
\hline
\bf Componente & \bf Especificação \\ \hline
Processador & [Modelo e frequência] \\ \hline
Memória RAM & [Quantidade] GB \\ \hline
GPU & [Modelo] ou N/A \\ \hline
Sistema Operacional & [Nome e versão] \\ \hline
Python & [Versão] \\ \hline
\end{tabularx}
\caption{Especificações do ambiente computacional}
\label{tab:hardware}
\end{table}

\subsection{Hiperparâmetros dos Modelos}

\subsection{Configuração do Aprendizado Federado}

\subsection{Experimentos Realizados}

\subsection{Métricas de Avaliação}

\section{RESULTADOS}

\subsection{Resultados em Cenário IID}

\subsubsection{Comparação de Modelos}

\subsubsection{Comparação de Estratégias}

\subsection{Resultados em Cenário non-IID}

\subsubsection{Comparação de Modelos}

\subsubsection{Impacto da Heterogeneidade}

\subsection{Análise de Convergência}

\subsection{Impacto da Integração SDN}

\subsection{Análise Estatística}

\subsubsection{Teste de Friedman}

\subsubsection{Teste Post-hoc de Nemenyi}

\section{DISCUSSÃO}

\subsection{Interpretação dos Resultados}

\subsection{Comparação com Estado da Arte}

\subsection{Limitações do Trabalho}

\subsection{Implicações Práticas}

\section{CONCLUSÕES E TRABALHOS FUTUROS}

\subsection{Conclusões}

\subsection{Contribuições}

\subsection{Trabalhos Futuros}


% REFERÊNCIAS
\renewcommand{\refname}{\centering REFERÊNCIAS}
\addcontentsline{toc}{section}{REFERÊNCIAS}
\begin{thebibliography}{99}
% Referências em formato ABNT

% Exemplo de artigo
\bibitem{exemplo2020}AUTOR, A.B.; COAUTOR, C.D. 
Título do artigo. {\bf Nome da Revista}, v.10, n.2, 
p.45-62, 2020.

% Exemplo de livro
\bibitem{exemplo_livro}AUTOR, Nome. 
{\bf Título do Livro}. Editora, Cidade, Ano.

% Exemplo de conferência
\bibitem{exemplo_conf}AUTOR, A.; COAUTOR, B.
Título do artigo. {\bf Nome da Conferência}, 
p.100-110, 2021.

% Adicionar referências conforme necessário

\end{thebibliography}

% ANEXOS
% Anexos (opcional)

% Descomentar se houver anexos

% \section*{\centering{ANEXO A – CÓDIGO FONTE}}

% \subsection*{Implementação do Cliente Flower}
% \begin{lstlisting}[language=Python]
% # Código aqui
% \end{lstlisting}

% \section*{\centering{ANEXO B – DOCUMENTAÇÃO ADICIONAL}}


\end{document}
